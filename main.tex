\documentclass[aspectratio=169,11pt]{beamer}

\setbeamertemplate{navigation symbols}{}
\useinnertheme[shadow=true]{rounded}
\useoutertheme{infolines}
\setbeamertemplate{footline}[frame number]
\definecolor{purpleheart}{RGB}{078, 042, 132}
\setbeamercolor{title}{fg=white,bg=purpleheart}
\setbeamercolor{palette tertiary}{fg=white,bg=purpleheart}
\setbeamercolor{frametitle}{fg=purpleheart,bg=gray!0}
\setbeamercolor{button}{bg=purpleheart}
\setbeamertemplate{itemize item}[circle]
\setbeamertemplate{enumerate item}[default]
\setbeamertemplate{itemize subitem}[triangle]
\setbeamercolor{itemize item}{fg=purpleheart}
\setbeamercolor{itemize subitem}{fg=purpleheart}
\setbeamercolor{enumerate item}{fg=purpleheart}
\setbeamercolor{part title}{fg=white,bg=purpleheart}
\setbeamertemplate{section page}
{
    \begin{centering}
    \begin{beamercolorbox}[wd=\paperwidth,sep=12pt,center]{part title}
    \usebeamerfont{section title}\insertsection\par
    \end{beamercolorbox}
    \end{centering}
}
\setbeamertemplate{title page}[default][rounded=true]
\usepackage{graphicx}
\usepackage{booktabs}
\usepackage{amsthm}
\usepackage{amssymb}
\usepackage{amsmath} 
\usepackage{bbm}
\usepackage{array}
\usepackage{hyperref}
\usepackage{appendixnumberbeamer}
\newcommand{\beginbackup}{
   \newcounter{framenumbervorappendix}
   \setcounter{framenumbervorappendix}{\value{framenumber}}
   \setbeamertemplate{footline}
   {
     \leavevmode%
     \hline
     box{%
       \begin{beamercolorbox}[wd=\paperwidth,ht=2.25ex,dp=1ex,right]{footlinecolor}%
%         \insertframenumber  \hspace*{2ex} 
       \end{beamercolorbox}}%
     \vskip0pt%
   }
 }
\newcommand{\backupend}{
   \addtocounter{framenumbervorappendix}{-\value{framenumber}}
   \addtocounter{framenumber}{\value{framenumbervorappendix}} 
}


\begin{document}
\title[Beamer]{A Beamer Template}
\author[Name]{\textbf{Your Name} \\Your Institution}

\begin{frame}[plain,noframenumbering]
\titlepage
\end{frame}

\section{Introduction}
\begin{frame}{Motivation}
\begin{itemize}
    \item Getting a mortgage is an important household financial decision.
    \begin{itemize}
        \item Loan size: \$9.56 trillion in Dec. 2019 (68\% of total household debt)
        \item Loan term: 7-10 years (the average lifespan)
        \item Prevalence: 42\% of families held debt secured by a primary residence in 2019
    \end{itemize}
    \pause
    \item Getting a mortgage is a complicated transaction.
    \begin{itemize}
        \item Lengthy process: conventional-57 days and FHA-62 days in 2021
        \item Hard to understand: mortgage disclosures (11th grade) vs. reading ability (8th grade)
    \end{itemize}
    \pause 
    \item Lenders exploit borrowers’ confusion and mislead them.\\ {\footnotesize \color{darkgray} (Agarwal et al., 14; Agarwal, Ben-David \& Yao, 17; Di Maggio, Kermani \& Korgaonkar, 19)}
    \pause
    \item \textbf{Borrowers with limited English proficiency (LEP) will encounter more barriers.}
\end{itemize}
\end{frame}

\begin{frame}{What is LEP}
\begin{itemize}
    \item \textbf{Definition}: speaking English less than ``very well''
    \item \textbf{Size}: 25 million LEP people in 2019 (8\% of the total population)
    \item<2-> \textbf{Primary language}: Spanish, Chinese, Korean, Vietnamese, Tagalog
\end{itemize}
\begin{figure}
\centering
\includegraphics<1>[height=52.5mm]{chai.png}%
\includegraphics<2->[height=52.5mm]{lepshare_hispanic.png}%
\end{figure}    
\end{frame}

\begin{frame}{This Paper}
\textbf{Research Question}: 

How language frictions affect LEP borrowers in the mortgage market on their
\begin{itemize}
    \item application experiences
    \item borrowing costs
    \item credit access
\end{itemize}
\pause

\end{frame}

\begin{frame}{Main Results}
\begin{itemize}
    \item Definition: speaking English less than ``very well''
    \item Size: 25 million LEP people in 2019 (8\% of the total population)
    \item Primary language: Spanish, Chinese, Korean, Vietnamese, Tagalog
\end{itemize}
\end{frame}

\begin{frame}{Related Literature}
\begin{itemize}
    \item Price dispersion in consumer finance markets
    \begin{itemize}
        \item Various markets: {\footnotesize \color{darkgray}Gurun et al., 16; Argyle et al., 20; Stango \& Zinman, 16}
        \item 
    \end{itemize}
    \item Real effects of government intervention in credit markets
    \begin{itemize}
        \item Earnings: {\footnotesize \color{darkgray} McManus et al., 83; Tainer, 88; Chiswick, 91; Zavodny, 00; Dustmann \& Fabbri, 03}
        \item Marriage, fertility, and health: {\footnotesize \color{darkgray}Bleakley \& Chin, 10; Guven \& Islam, 15}
    \end{itemize}
    \item Effects of English ability
    \begin{itemize}
        \item Earnings: {\footnotesize \color{darkgray} McManus et al., 83; Tainer, 88; Chiswick, 91; Zavodny, 00; Dustmann \& Fabbri, 03}
        \item Marriage, fertility, and health: {\footnotesize \color{darkgray}Bleakley \& Chin, 10; Guven \& Islam, 15}
    \end{itemize}
\end{itemize}
\end{frame}

\begin{frame}{Outline for Today}
\begin{columns}[T] % align columns
\begin{column}{.65\textwidth}
    \begin{enumerate}
        \item Data
        \item LEP borrowers in the mortgage market
        \item Policy shock: FHFA Language Access Plan
        \item Empirical strategy: triple-difference
        \item Results
        \begin{itemize}
            \item NSMO
            \item Aggregate Data
            \item $\text{HMDA}^+$
        \end{itemize}
        \item Conclusion
    \end{enumerate}
\end{column}
\begin{column}{.35\textwidth}
    \resizebox{\linewidth}{!}{
      \includegraphics{fig/chai.png}
    }
\end{column}
\end{columns}
\end{frame}


\section{Data}
\begin{frame}[plain,noframenumbering]
\sectionpage
\end{frame}

\begin{frame}{Data Sources}
\begin{itemize}
    \item \textbf{NSMO}
    \item \textbf{HMDA}
    \item \textbf{GSE}
    \item 
\end{itemize}    
\end{frame}

\section{Descriptive Analysis}
\begin{frame}[plain,noframenumbering]
\sectionpage
\end{frame}

\begin{frame}{LEP Status in NSMO}
\begin{columns}[T] % align columns
\begin{column}{.55\textwidth}
    {
      \includegraphics<1>[height=70mm,width=72mm]{nsmo_lep.jpg}%
      \includegraphics<2>[height=60mm,width=88mm]{share_year.pdf}%
      \includegraphics<3>[height=60mm,width=88mm]{education.pdf}%
      \includegraphics<4>[height=60mm,width=88mm]{income.pdf}%
      \includegraphics<5>[height=60mm,width=88mm]{fico.pdf}%
    }
\end{column}
\begin{column}{.45\textwidth}
    \begin{itemize}
        \item About 10\% are LEP borrowers
        \item Servicers in Dallas averaged having 15\% of LEP borrowers as customer
        \pause
        \item Display an increasing trend
        \pause 
        \item Demographic differences as expected
        \begin{itemize}
            \item Education
            \pause 
            \item Income
            \pause
            \item FICO score
        \end{itemize}
    \end{itemize}
\end{column}
\end{columns}    
\end{frame}

\begin{frame}{Differences in Mortgage Application Experiences}
\vspace{-15.5pt}
\begin{equation} \label{eq:ols}
    y_{it} = \alpha + \textcolor{red}{\beta LEP_i} + \gamma X_{i} + \delta_t + \epsilon_{it}
\end{equation}
\pause
\centerline{
\scalebox{0.85}{
\begin{tabular}{l<{\onslide<2->}c<{\onslide<3->}c<{\onslide<4->}c<{\onslide<5->}c<{\onslide<6->}c<{\onslide}} \toprule
Dependent variable & \multicolumn{5}{c}{$\mathbbm{1}$(concern about qualifying for a mortgage)}\\ 
 & (1) & (2) & (3) & (4) & (5)\\
 \midrule
LEP & 0.102*** & 0.100*** & 0.064*** & 0.058*** & 0.059*** \\
& (0.009) & (0.009) & (0.008) & (0.008) & (0.008) \\
 \midrule
 D.V. mean (LEP) & \multicolumn{5}{c}{0.243}   \\\midrule
Observations & 37,720 & 37,720 & 37,720 & 37,720 & 37,720 \\
Quarter FEs &  & \checkmark & \checkmark & \checkmark & \checkmark \\
Tract type FEs &  & \checkmark & \checkmark & \checkmark & \checkmark \\
Race and ethnicity &  &  & \checkmark & \checkmark & \checkmark \\
Gender &  &  & \checkmark & \checkmark & \checkmark \\
Education &  &  & \checkmark & \checkmark & \checkmark \\
Additional demo. controls &  &  & \checkmark & \checkmark & \checkmark \\
Risk FEs (FICO $\times$ LTV) &  &  &  & \checkmark & \checkmark \\
Loan controls &  &  &  &  & \checkmark \\
\bottomrule
\end{tabular}
}}
\end{frame}

\begin{frame}{Differences in Mortgage Application Perceptions and Experiences}
\begin{center}
\textbf{When you began the process of getting this mortgage, \\ how familiar were you with each of the following?}   
\end{center}
\begin{columns}[T] % align columns
\begin{column}{.5\textwidth}
    \resizebox{\linewidth}{!}{
      \includegraphics{beforeknow_raw.png}
    }
\end{column}
\begin{column}{.5\textwidth}
    \resizebox{\linewidth}{!}{
      \includegraphics{beforeknow_reg.png}
    }
\end{column}
\end{columns}
\end{frame}

\begin{frame}{Differences in Mortgage Application Perceptions and Experiences}
\begin{center}
\textbf{\footnotesize When you began the process of getting this mortgage, how familiar were you with each of the following?}   
\end{center}
\begin{figure}
\centering
\includegraphics[height=60mm]{beforeknow_reg.png}
\end{figure} 
\end{frame}

\begin{frame}[label=search]{Differences in Search Behavior}
\centerline{\scalebox{0.85}{
\begin{tabular}{lccccc} 
\toprule
Dependent variable & \multicolumn{2}{c}{Number of lenders} & \multicolumn{3}{c}{Why apply to multiple lenders?}\\ \cmidrule[0.6pt](lr){2-3} \cmidrule[0.6pt](lr){4-6}
 & \begin{tabular}[c]{@{}c@{}}seriously\\ considered\end{tabular} & applied to & \begin{tabular}[c]{@{}c@{}}find better \\loan terms\end{tabular} & \begin{tabular}[c]{@{}c@{}}concern over\\ qualification\end{tabular} & \begin{tabular}[c]{@{}c@{}}learn\\ information\end{tabular} \\ 
 & (1) & (2) & (3) & (4) & (5) \\
\midrule
LEP & -0.065*** & -0.024** & 0.016 & 0.105*** & 0.075*** \\
 & (0.015) & (0.012) & (0.017) & (0.020) & (0.021) \\
\midrule
 LEP mean & 1.643 & 1.296 &  0.821 & 0.407 & 0.425  \\
 Non-LEP mean & 1.719 & 1.303 & 0.822 & 0.270 & 0.319 \\ \midrule
Observations & 37,720 & 37,720 & 8,569 & 8,569 & 8,569 \\
Quarter FEs & \checkmark & \checkmark & \checkmark & \checkmark & \checkmark \\
Tract type FEs & \checkmark & \checkmark & \checkmark & \checkmark & \checkmark \\
Demographic controls& \checkmark & \checkmark & \checkmark & \checkmark & \checkmark \\
Risk FEs & \checkmark & \checkmark & \checkmark & \checkmark & \checkmark \\
Loan controls & \checkmark & \checkmark & \checkmark & \checkmark & \checkmark \\
\bottomrule
\end{tabular}
}}
\hfill \hyperlink{search_demo}{\beamergotobutton{Demographic coefficients}}
\end{frame}

\section{Policy Shock}
\begin{frame}[plain,noframenumbering]
\sectionpage
\end{frame}

\begin{frame}{FHFA Language Access Plan}
\begin{figure}
\centering
\includegraphics[height=50mm]{policyshock.png}
\end{figure}       
\end{frame}

\begin{frame}{Google Trends}
\begin{figure}
\centering
\includegraphics[height=68mm]{google_trends.png}
\end{figure}       
\end{frame}

\section{Empirical Strategy}
\begin{frame}[plain,noframenumbering]
\sectionpage
\end{frame}

\begin{frame}{Triple-Difference Illustration}
\begin{figure}
\centering
\includegraphics[height=75mm]{redopaperwork_ddd.png}
\end{figure}          
\end{frame}

\begin{frame}{Triple-Difference Specification}
\begin{multline} \label{eq:ddd}
    y_{it} = \alpha + \beta_0 LEP_i + \beta_1 Hispanic_i + \beta_2 LEP_i \times Hispanic_i + \beta_3 LEP_i \times Post_t \\+ \beta_4 Hispanic_i \times Post_t + \textcolor{red}{\beta_5 LEP_i \times Hispanic_i \times Post_t} + \gamma X_{it} + \delta_t + \epsilon_{it}.    
\end{multline}    
\begin{itemize}
    \item Drop Asian borrowers (Chinese translations added in 2019)
\end{itemize}
\end{frame}

\section{Empirical Results}
\begin{frame}[plain,noframenumbering]
\sectionpage
\end{frame}

\begin{frame}{Results Using NSMO}
    
\end{frame}

\section{Conclusion}
\begin{frame}{Conclusion}
    
\end{frame}

\section{Thanks!}
\begin{frame}[plain,noframenumbering]
\sectionpage
\end{frame}

\section{Appendix}
\appendix
\begin{frame}[label=search_demo]{Differences in Search Behavior}
\centerline{\scalebox{0.85}{
\begin{tabular}{lccccc} 
\toprule
Dependent variable & \multicolumn{2}{c}{Number of lenders} & \multicolumn{3}{c}{Why apply to multiple lenders?}\\ \cmidrule[0.6pt](lr){2-3} \cmidrule[0.6pt](lr){4-6}
 & \begin{tabular}[c]{@{}c@{}}seriously\\ considered\end{tabular} & applied to & \begin{tabular}[c]{@{}c@{}}find better \\loan terms\end{tabular} & \begin{tabular}[c]{@{}c@{}}concern over\\ qualification\end{tabular} & \begin{tabular}[c]{@{}c@{}}learn\\ information\end{tabular} \\ 
 & (1) & (2) & (3) & (4) & (5) \\
\midrule
LEP & -0.065*** & -0.024** & 0.016 & 0.105*** & 0.075*** \\
 & (0.015) & (0.012) & (0.017) & (0.020) & (0.021) \\
\midrule
 LEP mean & 1.643 & 1.296 &  0.821 & 0.407 & 0.425  \\
 Non-LEP mean & 1.719 & 1.303 & 0.822 & 0.270 & 0.319 \\ \midrule
Observations & 37,720 & 37,720 & 8,569 & 8,569 & 8,569 \\
Quarter FEs & \checkmark & \checkmark & \checkmark & \checkmark & \checkmark \\
Tract type FEs & \checkmark & \checkmark & \checkmark & \checkmark & \checkmark \\
Demographic controls& \checkmark & \checkmark & \checkmark & \checkmark & \checkmark \\
Risk FEs & \checkmark & \checkmark & \checkmark & \checkmark & \checkmark \\
Loan controls & \checkmark & \checkmark & \checkmark & \checkmark & \checkmark \\
\bottomrule
\end{tabular}
}}
\hfill \hyperlink{search}{\beamergotobutton{Back}}    
\end{frame}
\end{document}